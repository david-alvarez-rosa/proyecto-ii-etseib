% -*- TeX-master: "Informe.tex" -*-


\section{Árbol de directorios}

A continuación se muestra en la figura \ref{fig:directorios} el árbol de
directorios del Servidor Web, con todos los diferentes archivos que se han
creado. \\

\begin{figure}[H]
  \centering
  \scalebox{.93}{% -*- TeX-master: "Informe.tex" -*-

\begin{minipage}{.4\textwidth}
  \dirtree{%
    .1 \myFolder{Servidor Web}.
    .2 \myFolder{cgi-bin}.
    .3 {* El contenido está $\longrightarrow$}.
    %
    .2 \myFolder{jugar}.
    .3 \myFolder{ai}.
    .4 \myPhp{index}.
    .3 \myFolder{per}.
    .4 \myPhp{index}.
    %
    .3 \myCss{animacion}.
    .3 \myPhp{animacion}.
    .3 \myJs{animacion}.
    .3 \myJs{confeti}.
    .3 \myPhp{controles}.
    .3 \myPhp{data}.
    .3 \myPhp{index}.
    .3 \myPhp{tablero}.
    %
    .2 \myFolder{presentaciones}.
    .3 \myPhp{index}.
    .3 \myFolder{p1}.
    .4 \myPhp{index}. .4 \myPdf{p1}. .4 \myZip{p1}.
    .3 \myFolder{p2}.
    .4 * Análogo a p1 $\uparrow$.
    % .4 \myPhp{index}. .4 \myPdf{p1}. .4 \myZip{p1}.
    .3 \myFolder{p3}.
    .4 * Análogo a p1 $\uparrow$.
    % .4 \myPhp{index}. .4 \myPdf{p1}. .4 \myZip{p1}.
    .3 \myPhp{index}.
    %
    .2 \myFolder{proyecto}.
    .3 \myFolder{codigo}. % TODO: Revisar esta carpeta.
    .4 \myPhp{index}. .4 \myZip{TresEnRaya}.
    .3 \myFolder{documentacion}. % TODO: Revisar esta carpeta.
    .4 \myPhp{index}. .4 \myPdf{documentacion}. .4 \myZip{documentacion}.
    .3 \myPhp{index}.
    %
    .2 \myHtml{footer}.
    .2 \myPhp{index}.
    .2 \myCss{layout}.
    .2 \myImg{logo\_ETSEIB} \imgInline{Logo_ETSEIB.png}.
    .2 \myImg{logo\_UPC} \imgInline{Logo_UPC.png}.
    .2 \myHtml{navbar}.
    .2 \myCss{tablero}.
  }
\end{minipage}
\hspace{2.25cm}
\begin{minipage}{.4\textwidth}
  \dirtree{%
    .1 \myFolder{cgi-bin}.
    .2 \myFolder{movimiento}.
    .3 \myPy{main}.
    .3 \myPy{plano}.
    .3 \myPy{servos}.
    .2 \myFolder{estrategia}.
    .3 \myPy{basica}.
    .3 \myPy{cfg}.
    .3 \myPy{equivalencias}.
    .3 \myPy{main}.
    .2 \myPy{main}.
    .2 \myPy{main2}.
  }
\end{minipage}
}
  \caption{Árbol de directorios.}
  \label{fig:directorios}
\end{figure}


\section{Código más representativo}

A continuación se recoge el código desarrollado considerado más representativo
en este proyecto. Se puede consultar el código completo en la
\href{https://alvarezrosa.com/proyecto/proyecto/}{web} del proyecto.

\subsection{Estrategia} \label{sec:cod_est}
Aquí se recoge todo el codigo desarrollado en Python \imgInline{Logo_Python.png}
relacionado con la estrategia de juego del Tres en Raya. Este código en el
servidor se encuentra (de acuerdo con la figura \ref{fig:directorios} de la
página \pageref{fig:directorios}) en el siguiente directorio: \\

\dirtree{%
  .1 \myFolder{Servidor Web}.
  .2 \myFolder{cgi-bin}.
  .3 \myFolder{estrategia}.
  .4 \myPy{basica}.
  .4 \myPy{cfg}.
  .4 \myPy{equivalencias}.
  .4 \myPy{main}.
}

\subsubsection*{Estrategia Básica (\textit{basica.py})}
\pycode[label = basica.py]{../Codigo/estrategia/basica.py}
\subsubsection*{Auxiliar (\textit{cfg.py})}
\pycode[label = cfg.py]{../Codigo/estrategia/cfg.py}
\subsubsection*{Equivalencias (\textit{equivalencias.py})}
\pycode[label = equivalencias.py]{../Codigo/estrategia/equivalencias.py}
\subsubsection*{Programa principal (\textit{main.py})}
\pycode[label = main.py]{../Codigo/estrategia/main.py}


\subsection{Movimiento} \label{sec:cod_mov}
Aquí se recoge todo el codigo desarrollado en Python \imgInline{Logo_Python.png}
relacionado con el movimiento del brazo robótico. Este código en el servidor se
encuentra (de acuerdo con la figura \ref{fig:directorios} de la página
\pageref{fig:directorios}) en el siguiente directorio: \\

\dirtree{%
  .1 \myFolder{Servidor Web}.
  .2 \myFolder{cgi-bin}.
  .3 \myFolder{movimiento}.
  .4 \myPy{main}.
  .4 \myPy{plano}.
  .4 \myPy{servos}.
}

\subsubsection*{Programa principal (\textit{main.py})}
\pycode[label = main.py]{../Codigo/movimiento/main.py}
\subsubsection*{Plano (\textit{plano.py})}
\pycode[label = plano.py]{../Codigo/movimiento/plano.py}
\subsubsection*{Servomotores (\textit{servos.py})}
\pycode[label = servos.py]{../Codigo/movimiento/servos.py}


\subsection{Programa principal}
Aquí se recoge todo el codigo desarrollado en Python \imgInline{Logo_Python.png}
encargado de fusionar/coordinar el código de las secciones anteriores
(\ref{sec:cod_est} y \ref{sec:cod_mov}). Este código en el servidor se
encuentra (de acuerdo con la figura \ref{fig:directorios} de la página
\pageref{fig:directorios}) en el siguiente directorio: \\

\dirtree{%
  .1 \myFolder{Servidor Web}.
  .2 \myFolder{cgi-bin}.
  .3 \myPy{main}.
  .3 \myPy{main2}.
}

\subsubsection*{Programa principal 1 (\textit{main1.py})}
\pycode[label = main.py]{../Codigo/mainWeb.py}
\subsubsection*{Programa principal 2 (\textit{main2.py})}
\pycode[label = main2.py]{../Codigo/mainWeb2.py}


\subsection{Animación}
Aquí se recoge el código desarrollado en JavaScript \imgInline{Logo_JS.png}
que hace posibles las animaciones del brazo robótico virtual que se muestra en
la web. También se añade su corrrespondiente HTML \imgInline{Logo_HTML.png} y
CSS \imgInline{Logo_CSS.png}. Este código en el servidor se encuentra (de
acuerdo con la figura \ref{fig:directorios} de la página
\pageref{fig:directorios}) en el siguiente directorio: \\

\dirtree{%
  .1 \myFolder{Servidor Web}.
  .2 \myFolder{jugar}.
  .3 \myHtml{animacion}.
  .3 \myJs{animacion}.
}

\subsubsection*{Animación HTML (\textit{animacion.php})}
\phpcode[label = animacion.php]{../Codigo/Servidor\ Web/jugar/animacion.php}
\subsubsection*{Animación JS (\textit{animacion.js})}
\jscode[label = animacion.js]{../Codigo/Servidor\ Web/jugar/animacion.js}


\subsection{Servidor}
El resto de la parte de archivos del servidor han sido desarrollados en diversos
lenguajes, como son PHP \imgInline{Logo_PHP.png}, HTML
\imgInline{Logo_HTML.png}, CSS \imgInline{Logo_CSS.png} y JavaScript
\imgInline{Logo_JS.png}. En este caso, debido a la extensión, solo se mostrarán
los archivos considerados más representativos. Más concretamente se mostrarán
los recogidos en el siguiente árbol (de acuerdo con la figura
\ref{fig:directorios} de la página \pageref{fig:directorios}). \\

\dirtree{%
  .1 \myFolder{Servidor Web}.
  %
  .2 \myFolder{jugar}.
  .3 \myFolder{ai}.
  .4 \myPhp{index}.
  .3 \myPhp{controles}.
  .3 \myPhp{data}.
  .3 \myPhp{tablero}.
  %
  .2 \myHtml{footer}.
  .2 \myHtml{navbar}.
}

\vspace{.5cm}
El contenido de los archivos a continuación se muestra siguiendo el mismo orden
en el que aparecen en el árbol anterior.

\subsubsection*{Página principal jugar/ai/ (\textit{index.php})}
\phpcode[label = index.php]{../Codigo/Servidor\ Web/jugar/ai/index.php}
\subsubsection*{Controles (\textit{controles.php})}
\phpcode[label = controles.php]{../Codigo/Servidor\ Web/jugar/controles.php}
\subsubsection*{Tabla datos diversos (\textit{data.php})}
\phpcode[label = data.php]{../Codigo/Servidor\ Web/jugar/data.php}
\subsubsection*{Tablero del Tres en Raya (\textit{tablero.php})}
\phpcode[label = tablero.php]{../Codigo/Servidor\ Web/jugar/tablero.php}
\subsubsection*{Pie de página (\textit{footer.html})}
\htmlcode[label = footer.html]{../Codigo/Servidor\ Web/footer.html}
\subsubsection*{Barra de navegación (\textit{navbar.html})}
\htmlcode[label = navbar.html]{../Codigo/Servidor\ Web/navbar.html}
